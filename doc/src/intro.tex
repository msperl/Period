\section{What is \periodname?}

\period is a program
to search for and fit sinusoidal patterns within a time series
of data in which one suspects periodic behavior.

This kind of data sets commonly occur in observational science such as
astronomy (light curves of stars), 
meteorology (daily temperature measurements)
or physics.

\period has been written to solve problems of large 
astronomical data sets containing huge gaps.
But as mentioned earlier, in other fields such data 
also occur and thus often a similar approach can be taken.
So even though most examples are taken from astronomical data 
the basics will apply to all kind of data, even if these are 
from an entirely different field.

Some commonly used techniques to solve such problems are:
\myitemize{
\item Minimization of residuals of sinusoidal fits to the data
\item Fourier analysis and Fast Fourier analysis
\item Clean
\item Wavelet
\item Maximum entropy
\item Phase dispersion minimization
\item etc.
}

Many thanks to Julian Smart and Markus Helmholz for the wxWindows GUI library.

%%% Local Variables: 
%%% mode: latex
%%% TeX-master: "period98"
%%% End: 
