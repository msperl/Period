\section{Microsoft Windows95 and Windows NT}%
\label{Install_Windows95NT}

To install \period on Windows95 or Windows NT you have to
get the file
\urlref{\periodnt}{\periodntftp}
and start it.
This will lead you through the setup procedure 
and finally installs \period on the start menu.

%%%%%%%%%%%%%%%%%%%%%%%%%%%%%%%%%%%%%%%%%%%%%%%%%%%%%%%%%%%%%%%%%%%%%%%%%%%%%%%
\section{Unix - Install precompiled}%
\label{Install_Unix_precompiled}

The following UNIX executables are currently available:
\myitemize{\periodbinaries}

To review the latest list of compilation of 
\period executable packages please have a look at 
the \urlref{\periodname homepage}{\periodhome}.

To install, simply type the command:
{\tt sh \periodfull-<OSTYPE>}
as this is a shell script.
You will be asked a few questions about where to install the relevant files.
First, where all the files needed for \period should be installed, and then
where the binary itself should be installed.

When the program has finished, \period is ready to use and can be 
started with the command: {\tt \periodname}.

%%%%%%%%%%%%%%%%%%%%%%%%%%%%%%%%%%%%%%%%%%%%%%%%%%%%%%%%%%%%%%%%%%%%%%%%%%%%%%%
\newpage
\section{Unix - Install from source code}%
\label{Install_Unix_source}

To install \period from source code you must have the 
following programs installed on your computer:
\myitemize{
\item C++ compiler (preferably GNU-gcc 2.7.2 or later, others may not work)
\item gzip
\item possibly GNU-make
}

%%%%%%%%%%%%%%%%%%%%%%%%%%%%%%%%%%%%%%%%%%%%%%%%%%%%%%%%%%%%%%%%%%%%%%%%%%%%%%%
\subsection{What files are needed}%
\label{Get_SRCS}

Please get all of the following files:
\myitemize{
\item wxxt-GUI-library:
 \urlref{\wxxtlib\\}{\wxxtliburl}
\item wxxt-GUI-library utils:
 \urlref{\wxxtutils\\}{\wxxtutilsurl}
\item \periodname sources:
 \urlref{\periodsrc\\}{\periodsrcurl}
}

%%%%%%%%%%%%%%%%%%%%%%%%%%%%%%%%%%%%%%%%%%%%%%%%%%%%%%%%%%%%%%%%%%%%%%%%%%%%%%%
\subsection{Compiling \periodname}%
\label{Compiling_Period}

\myitemize{
\item Extract files for compilation
        \myitemize{
        \item extract the file {\bf \wxxtlib} with the command:\\
                {\tt zcat \wxxtlib | tar xf - }
        \item enter the directory {\bf \wxxtbase} with:\\
                {\tt cd \wxxtbase}
        \item extract the file \wxxtutils\space with the command:\\
                {\tt zcat ../\wxxtutils | tar xf - }
        \item enter the directory {\bf user} with:\\
                {\tt cd user}
        \item extract the file {\bf \periodsrc} with the command:\\
                {\tt zcat ../../\periodsrc | tar xf - }
        \item enter the directory {\bf \periodname} with:\\
                {\tt cd \periodname}
        \item go back to the {\bf main} directory with:\\
                {\tt cd ../..}
        }
\item Configure your system with the command:\\
        {\tt ./configure}\\
        If this does complain about something, then please
        try to fix the situation according to the messages
        which you receive.\\
        The most probable events may be that the version
        of {\bf make} installed on your machine does not support
        some features needed to compile or that your shell does not define 
        the system variable OSTYPE. 
\item Compile the relevant components with the command:\\
        {\tt make all SRC='' src utils/wxTab utils/wxHelp user/\periodnamens''}\\
        in the case of compile or link problems that mention something
        about xpm please retry to configure your system with the
        additional switch:
        {\tt $--$without-xpm}, then type {\tt make clean} 
        and then start compiling anew.
\item Create an installable executable package of \period with the command:\\
        {\tt {./\periodnamens}Package}\\
        This will create a file called \periodfull-$<$OSTYPE$>$.
        This is a final distributable file for your system.
\item Install the package as explained to you in:
        \helpref{Installing precompiled binaries}{Install_Unix_precompiled}. 
}

After some tests with the newly compiled \period we would welcome if you
could send us a copy of your compiled \period.
To do so please upload to the 
\urlref{\periodname incoming directory}{\periodftpincoming} 
and send a mail to \periodmail, 
so that we may include it, along with other binary distributions 
for the convenience of other potential users.

%%% Local Variables: 
%%% mode: latex
%%% TeX-master: "period98"
%%% End: 
