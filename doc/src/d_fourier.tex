\section{Fourier}%
\label{fourier.detailed}

%%%%%%%%%%%%%%%%%%%%%%%%%%%%%%%%%%%%%%%%%%%%%%%%%%%%%%%%%%%%%%%%%%%%%%%%%%%%%%%
\subsection{Fourier folder}%
\label{fourier.folder}

\begin{figure}[h]
$$\image{0cm;0cm}{FFolder.eps}$$
\caption{The ``Fourier'' folder}
\end{figure}

This folder shows all the necessary information that has to do with 
Fourier transformation.

The basic idea behind this folder is that different Fourier calculations 
can be kept in memory simultaneously and that the setting for each of these 
{\it Fourier sets} can be restored with a click of the mouse.

Because of this possibility there is a need for a {\bf Title} for 
identification by the user.
Then there are {\bf From} and {\bf To}, which define the frequency range,
for which the Fourier spectrum should be calculated.
As with the \helpref{fit module}{period.folder} there is the possibility
to use weighted data for the Fourier calculations.

 In the next line the Nyquist frequency
is displayed. This value should be a good
estimate for the upper limiting frequency due to the sampling pattern in
the currently selected time string.
This frequency value given does {\it not} depend only on the time base 
and the number of points in the currently selected time string, 
but uses a more sophisticated algorithm, estimating the average 
time gap between neighboring points, ignoring large gaps.

With the choice item {\bf Step rate} it is possible to change the accuracy 
of the Fourier calculation. This means, that the step at which the frequency 
range is sampled can be changed. {\bf High, medium and low} are good 
estimates (High uses most time to calculate).
With {\bf Custom} a user defined value can be chosen 
in the text item next to it.
For the other 3 possibilities the text item can not be edited and
will contain the value that would be used in this case.

Next the {\bf highest peak} line gives the highest peak that was 
found during the calculation with the coresponding frequency and amplitude.

With {\bf Calculations based on} it is possible to select different data types
for the calculation. These correspond to the values that can be displayed in
the \helpref{Time string graph}{timestring.graph} and
\helpref{Time string table}{timestring.table}%
. {\bf Spectral window }allows the calculation of the spectral window centered
at zero frequency for the currently selected time string.

The {\bf Compact} option allows to limit the - often very massive - output 
of Fourier, because with {\bf Peaks only} only the local minima and maxima
are kept.

The \button{Calculate} takes all the settings above and 
calculates a new Fourier spectrum with this. (See
\helpref{Fourier calculate}{fourier.calculate}
for more details.

In the list box below the \button{Calculate}, all the previous
Fourier calculations are listed with their title. Selecting one of these will
update all the values above to the values used for that specific
calculation.

With the \button{Delete} the currently selected Fourier calculation
can be removed from memory - only the log file entries will remain.
The \button{Rename} allows to change the title for the currently
selected Fourier calculation. To do so, the
\helpfigref{rename title dialog}{fourier.rename.dialog} is opened.
\begin{figure}[h]
$$\image{0cm;0cm}{FRename.eps}$$%
\caption{The ``rename title'' dialog}%
\label{fourier.rename.dialog}
\end{figure}

The \button{Display data} opens up the 
\helpref{Fourier data window}{fourier.table}
and allows examination of the output of the calculation.

Next the \button{Export} allows the output to be written to a file.
(See \helpref{Fourier export}{fourier.export} for details.)

Finally the \button{Display graph} opens up the
\helpref{Fourier graph window}{fourier.graph}

%%%%%%%%%%%%%%%%%%%%%%%%%%%%%%%%%%%%%%%%%%%%%%%%%%%%%%%%%%%%%%%%%%%%%%%%%%%%%%%
\subsection{Fourier calculate}%
\label{fourier.calculate}
When starting a Fourier calculation with the \button{Calculate} in
the \helpref{Fourier folder}{fourier.folder}%
, the main window will change
to represent the 
\helpfigref{current status of calculation}{fourier.calculate.dialog}%
. In this case it will give the percentage of the calculations completed.
Again the \button{Cancel} can stop the calculations.
\begin{figure}[h]
$$\image{0cm;0cm}{FCalculate.eps}$$%
\caption{The ``Fourier calculate'' window}%
\label{fourier.calculate.dialog}
\end{figure}

In the case of {\bf observed} or {\bf adjusted} values selected 
for calculations, the user will first be
asked if the average zero point should be subtracted with the
\helpref{zero point dialog}{fourier.zeropoint}%
.

After the calculation has finished, the
\helpfigref{include dialog}{fourier.include.dialog}
will show up and ask, if the currently found highest peak
should be included in the list of frequencies of the
\helpref{Fit module}{period.folder}%
.

\begin{figure}[h]
$$\image{0cm;0cm}{FInclude.eps}$$%
\caption{The ``include'' dialog}%
\label{fourier.include.dialog}
\end{figure}

%%%%%%%%%%%%%%%%%%%%%%%%%%%%%%%%%%%%%%%%%%%%%%%%%%%%%%%%%%%%%%%%%%%%%%%%%%%%%%%
\subsection{Zero point}%
\label{fourier.zeropoint}
The Fourier transform has the property, that if the mean amplitude
of the currently selected time string for calculation is {\it not Zero}
additional features in the low frequency range will show up.
This is due to this Zero point shift and the features showing up are 
comparable to a scaled spectral window centered at frequency 0.
This feature may even dominate the whole spectrum.

To overcome this problem \period will ask with the
\helpfigref{zero point dialog}{fourier.zeropoint.dialog}, if
it should subtract a (calculated) zero point, when the calculations
are based on {\bf Original} or {\bf Adjusted} data.
\begin{figure}[h]
$$\image{0cm;0cm}{FZeropoint.eps}$$%
\caption{The ``zero point'' dialog}%
\label{fourier.zeropoint.dialog}
\end{figure}

%%%%%%%%%%%%%%%%%%%%%%%%%%%%%%%%%%%%%%%%%%%%%%%%%%%%%%%%%%%%%%%%%%%%%%%%%%%%%%%
\subsection{Fourier export}%
\label{fourier.export}

\subsection{Fourier table}%
\label{fourier.table}
This \helpfigref{window}{fourier.table.window},
which shows up, when the \button{Display data} in the
\helpref{Fourier folder}{fourier.folder} has been pressed,
shows a table of {\bf frequencies, Fourier amplitudes and Fourier power},
of the currently active Fourier calculation.
\begin{figure}[h]
$$\image{0cm;0cm}{FTable.eps}$$%
\caption{The ``Fourier table'' window}%
\label{fourier.table.window}
\end{figure}

%%%%%%%%%%%%%%%%%%%%%%%%%%%%%%%%%%%%%%%%%%%%%%%%%%%%%%%%%%%%%%%%%%%%%%%%%%%%%%%
\subsection{Fourier graph}%
\label{fourier.graph}
The \helpfigref{Fourier graph window}{fourier.graph.window} will show up,
when the \button{Display Graph} in the
\helpref{Fourier folder}{fourier.folder} has been pressed.
It will show the graph of the currently active Fourier calculation.
\begin{figure}[h]
$$\image{0cm;0cm}{FGraph.eps}$$%
\caption{The ``Fourier graph'' window}%
\label{fourier.graph.window}
\end{figure}

Actually many of the menus are the same as for the
\helpref{time string graph}{timestring.graph}.

The only new feature is the \menu{display}, in which it is
possible to change the graph to display {\it power} instead of {\it amplitude}
with the \menuentry{Use power}.

%%%%%%%%%%%%%%%%%%%%%%%%%%%%%%%%%%%%%%%%%%%%%%%%%%%%%%%%%%%%%%%%%%%%%%%%%%%%%%%
\subsection{Noise}%
\label{fourier.noise}
Sadly observations are not perfect and usually contaminated with noise.
This noise may come from many sources: Observations, instrumentation,
the object itself, modes (frequencies) not yet found in the spectra, etc\ldots

As this noise is thus only pseudo-random there is no way to eliminate 
it during data reduction.
Thus all the noise produces some unpredictable pattern in the Fourier spectrum.
So it is possible, that some peaks in the Fourier spectrum are found that
are not present in the object observed.

But how can noise be separated from a real signal?
There is never a guarantee for a sure identification, 
but observational
(\urlref{Breger et al., 1993}{A\&A 271,482})
and numerical simulations
(\urlref{Kuschnig et al.,1997}{A\&A 328,544})
have shown, that the {\it ratio} between {\it signal} and {\it noise} in 
amplitude should at least be 4.0 to give good confidence.

So {\bf signal} is defined as either 
the amplitude in the {\bf Fourier spectrum}
or the amplitude of the {\bf least square fit solution}
for a certain peak, and {\bf noise} is defined as the average amplitude
in a close frequency range to the peak unde consideration
after subtracting the frequency with the 
\helpref{fit module}{period.folder}
and using the resulting residuals for the calculation of noise.
The \helpref{Fourier module}{fourier.folder} offers two options 
to calculate noise.
These are:
\helpref{Noise at frequency}{fourier.noisefrequency}
and
\helpref{Noise spectrum}{fourier.noisespectrum},
which both can be reached from the \menu{Fourier}.

%%%%%%%%%%%%%%%%%%%%%%%%%%%%%%%%%%%%%%%%%%%%%%%%%%%%%%%%%%%%%%%%%%%%%%%%%%%%%%%
\subsubsection{Noise at frequency}%
\label{fourier.noisefrequency}
When the \menuentry{Noise at frequency} in the \menu{Fourier}
has been selected, the
\helpfigref{noise at frequency dialog}{fourier.noisefrequency.dialog}
will show up.
\begin{figure}[h]%
\label{fourier.noisefrequency.dialog}
$$\image{0cm;0cm}{FNoise.eps}$$%
\caption{The ``noise at frequency'' dialog}%
\end{figure}

Most of the contents of this dialog are similar to that of the
\helpref{Fourier folder}{fourier.folder}.

The only difference is that the frequency at which the noise
calculations should be centered can be entered as well as
the extent of the range.

After the \button{OK} has been pressed and the calculation has finished
a window will pop up and show the result.

%%%%%%%%%%%%%%%%%%%%%%%%%%%%%%%%%%%%%%%%%%%%%%%%%%%%%%%%%%%%%%%%%%%%%%%%%%%%%%%
\subsubsection{Noise spectrum}%
\label{fourier.noisespectrum}
The noiselevel is not necessarily constant in the whole frequency spectrum.
The {\bf Fourier module} also supports the possibility to calculate a
{\bf noise spectrum}.

This is basically the same procedure as for
\helpref{noise at frequency}{fourier.noisefrequency}, but the calculation
is repeated at different parts of the frequency range.

When the \menuentry{Noise spectrum} in the \menu{Fourier}
has been selected, the
\helpfigref{noise spectrum dialog}{fourier.noisespectrum.dialog}
will show up.
\begin{figure}[h]
$$\image{0cm;0cm}{FNoiseSpectrum.eps}$$%
\caption{The ``noise spectrum'' dialog}%
\label{fourier.noisespectrum.dialog}
\end{figure}

The dialog is very similar to that of 
\helpref{noise at frequency}{fourier.noisefrequency}.
The main difference is that the frequency range {\bf from} and {\bf to}
has to be entered as well as the spacing between consecutive
noise values centered at frequencies of interest.
After the \button{OK} has been pressed and the calculation has finished,
a window will pop up and show the results.

%%% Local Variables: 
%%% mode: latex
%%% TeX-master: "period98"
%%% End: 
