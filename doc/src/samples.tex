\note{the numbers here are all taken as random as of now and need to be corrected.
The whole chapter should be reread with great care once the actual sample datasets
have appeared...}

This following example presents some of the more sophisticated possibilities 
that \period offers in a step by step manner.
Also some basic concepts of signal-processing and their 
application in \period will be discussed briefly.

This sample assumes that the basic principles of how to
use \period are well understood and that the user can
reproduce the example in
\helpref{Using \periodname}{HELP_HOWTOUSE}.

The problem presented here is an application
of \period on an extended astronomical dataset.

%%%%%%%%%%%%%%%%%%%%%%%%%%%%%%%%%%%%%%%%%%%%%%%%%%%%%%%%%%%%%%%%%%%%%%%%%%%%%%%%%%%%%%%%%%%%%%%%%

\section{Footprint of an astronomical time-series photometric data set}

In optical astronomical observations a continuous data set 
that is larger than 12 hours is very rare case.
The usual length of a ``run'' is most likely less than 12 hours,
depending on the star itself, the observatory and
season.

But this is not all that can happen during observations.
The technique used to observe a star also introduce
a pattern in time of how often a star will be observed.

Different techniques are used for different kinds of
stars and on the scientific goal for the current observations.
\myitemize{
\item {\bf high speed:} is used to observe stars which show
  frequencies that have periods of less than one hour, as
  long term stability of the instrumentation used may not be 
  assumed apriori. Thus sophisticated calibrations
  have to be used to extend to longer periods.
  The data for these kind of observations is mostly equidistant
  and usually at least one measurement is taken every 10 seconds.
  Still some gaps are introduced into the data that are mostly due 
  to clouds, instrumental problems or recalibration efforts.
  
  The usual instrumentation used in this setup is
  a {\it high-speed multichannel photometer}.

\item {\bf multiple star technique:} this technique does not allow
  for high frequency analysis with periods of less than 20 minutes,
  as the average gap in time between measurements is usually 7 to 10 
  minutes (depending on the observing program). This gap is introduced
  by the fact that multiple stars are observed once after the other with
  a {\it photometer}.
  Usually at least 3 stars and the sky background are measured one after the other.
  One of these 3 stars is the star of interest. The others are used as 
  calibration stars and should {\it not} show variability.
  As each star has first to be centered in the photometer aperture and then
  the integration can be started a measurement, the total time to
  measure {\it one} star is at least 90 seconds. Usually the stars
  are also observed in different {\bf filters}.
  
\item {\bf CCD photometer:} using a {\it CCD-camera} as acquisition device
  allows for a new hybrid-method of the both above mentioned, as the camera
  allows to measure all of the stars used in the multiple star technique at
  the same time, thus removing the need to reposition the telescope as well.
  The time in between measurements is currently (1997) close to one minute 
  and is improving dramatically with each year. On the other hand long term
  stability of the system is also given.

\item {\bf single measurements} for variable stars with long periods 
  (month to years) are also possible.
}

Usually time is granted for some telescopes for a few days at least to study
variable stars, so a ``run'' contains daily junks of measurements, which
are combined to a total time string.

%%%%%%%%%%%%%%%%%%%%%%%%%%%%%%%%%%%%%%%%%%%%%%%%%%%%%%%%%%%%%%%%%%%%%%%%%%%%%%%%%%%%%%%%%%%%%%%%%

\section{Analysis of a run}
For this example data from a 3 star technique observation has been used.
The length of the observation is \toedit days.
The time string {\tt observ.dat} in the {\it samples directory}
should be loaded and then can be examined
in detail using the \helpref{time string graph}{timestring.graph} option of \period.

The time string can now be separated in different sub strings representing different nights.
For this the \helpref{subdivide selection}{timestring.subdivide} option of \period can be used.
The settings should work for the data. Now even the different nights are colored differently
in the \helpref{time string graph window}{timestring.graph}.

\subsection{Fourier analysis}
\subsection{Spectral window}
\subsection{Significance of a frequency}

%%%%%%%%%%%%%%%%%%%%%%%%%%%%%%%%%%%%%%%%%%%%%%%%%%%%%%%%%%%%%%%%%%%%%%%%%%%%%%%%%%%%%%%%%%%%%%%%%

\section{Multi site campaign}
To decrease the large gaps in time the only possibility is to use
telescopes located all over the world. This means that all of the
telescopes are interconnected.
\myitemize{
\item \urlref{Delta Scuti Network (DSN)}{\DSNHOME},
\item \urlref{Whole Earth Telescope (WET)}{\WETHOME},
\item \urlref{Small Telescope Array with CCD Cameras(STACC)}{\STACCHOME}
}
are a few of the currently better known networks.

These telescopes located over all continents do not only
improve time resolution due to the telescopes situated in
different time zones, but also improves chances for good weather at
at least one site of the globe every day...

As the observations of the dataset loaded were made in \toedit an appropriate
label for that site shall be given by selecting {\it all} data points and
then using the \helpref{Combine substrings}{timestring.rename} to give the
selected time string an new common label of 
``\toedit'' in the attribute {\it observatory}.

Now a time string for observations made at additional observatories has 
to be appended. With the \menuentry{Import time string} in the 
\menu{time string} additional time strings can be loaded.
The time string {\tt sites-u.dat} in the {\it samples directory} is needed now.
This file also contains the appropriate labels for the different 
observatories, where the data has been taken. No changes in the 
\helpref{import format dialog}{timestring.import.dialog} have to be made.

Examining the whole data with \helpref{time string graph}{timestring.graph} option of \period.

The time string can now be separated in different substrings representing different nights.
For this the \helpref{subdivide selection}{timestring.subdivide} option of \period can be used.
The settings should work for the data. Now even the different nights are colored differently
in the \helpref{time string graph window}{timestring.graph}.

\subsection{Fourier analysis}

\subsection{Spectral window}

\subsection{Period analysis}

\subsection{Phase plot - noise level for different subsets}

\subsection{Using weighted data}

%%%%%%%%%%%%%%%%%%%%%%%%%%%%%%%%%%%%%%%%%%%%%%%%%%%%%%%%%%%%%%%%%%%%%%%%%%%%%%%%%%%%%%%%%%%%%%%%%

\section{Combining campaigns from different years}

\subsection{Fourier analysis}

\subsection{Spectral window}

\subsection{Period analysis}

\subsection{Phase plot}

\subsection{Amplitude variations}

%%%%%%%%%%%%%%%%%%%%%%%%%%%%%%%%%%%%%%%%%%%%%%%%%%%%%%%%%%%%%%%%%%%%%%%%%%%%%%%%%%%%%%%%%%%%%%%%%

\section{Other possible applications}
There are a lot of other applications for which \period can be very useful.
Some of then will be presented here.

\subsection{A fit to different filters simultaneously}
As has already been shown \period can create a best fit of amplitudes
for different subsets of a time string. But that is not all, as it allows 
to calculate {\bf amplitude and phase} variations at the same time.
This can be used to find a best fit to two substrings that 
represent different physical parameters of a certain object, 
but which share the same frequency.

One such example is the observation of a star in many different filters.
There are two possibilities to do so:
\myitemize{
\item {\bf ``Normal calculation'' of a fit:} For this a best fit for 
  {\bf all} parameters has to be calculated with one of the subsets.
  Then the other subset has to be selected and just Amplitudes and phases
  may be calculated anew (with the use of the \button{Calculate} in the ``Fit'' filer). 
  Then the amplitudes and phases for the different subsets may be 
  examined and variations can be found.
  
  \note{The drawback in this case is that the frequency does not get improved, 
  and that so the frequency may produce a best fit for one frequency, 
  but not a good one for the other. Cross checks with the other subsets used
  to fit the frequencies as well should also be made and care has to be taken.}

  In special cases the \button{Improve Special} in the ``Fit'' filer 
  may also be used if the phases should be kept fixed. 

\item {\bf Amplitude and phase variation:} With this option \period tries to improve
  a set of amplitudes and phases for each subset (filter) separately, but improving
  the frequency as a common variable.

  \note{The drawback with this method is that the calculation increases for the 
  most general case, 
  where for all frequencies amplitude and phase-variations are calculated, 
  the number of points will increase by a factor of 2 in the most optimistic case, 
  but the degrees of freedom will increase by a factor of 5/3 for two subsets.
  So, in the case that the number of points of the second subset are 
  less then 2/3 of the number of points then the first one, this does not
  mean numerical improvement of the fit. 
  The fit becomes more instable numerically than it has been before!
  In this case the first method should be used with the subset with the
  most points.

  This rational changes, when weighted data is used.
  In that case it is not easy to say, where the borderline would be...
  }
}

\subsection{Asymmetric light-curves}
The light curve of many stars show asymmetries from a sinusoidal
curve, some even bumps in the light curve, so \period can also
cope with this.

Fourier theory offers the general idea to use the harmonics 
of a frequency (base frequency times an integer as new frequency)
with a different amplitude and phase, to fit an arbitrary curve.

In the case of an asymmetric light curve, which can be best observed
with the \helpref{phase plot tool}{period.phase} tool of \period,
the Fourier spectrum will show at the frequency, twice the frequency,
etc. a peak, some of which may be very small, others may prove to be
significant.

A detailed description of how to enter harmonics as frequencies can
be found in the \helpref{Fit folder}{period.folder}.

\subsection{Frequency combination}
Linear theory also cannot explain frequency combinations.
That means that a frequency, that is a sum of two other frequencies,
is found to be significant in a light curve of a star.
No bulletproof physical reason has emerged yet, but 
{\it mode coupling} one possible explanation.

A detailed description of how to enter frequency combinations as frequencies can
be found in the \helpref{Fit folder}{period.folder}.

\note{Frequency combinations should {\bf only} be used when a frequency looks suspiciously
  like a sum of two others. That means that the difference between the independently
  computed value and the sum of the other two is {\it negligible}.
  In this case the use of frequency combination improves the stability of the 
  numerical fit, as the degrees of freedom decrease.} 


%%% Local Variables: 
%%% mode: latex
%%% TeX-master: "period98"
%%% End: 
