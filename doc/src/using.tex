As from the early design stages, the philosophy for \period had been
usability and scientific reproducibility. Therefore its central feature is 
its {\bf project file}. This offers the possibility to stop at a certain point 
and continue later - maybe even on a totally 
different operating system - as if the program had not been stopped.

Along with this comes the {\bf log feature}, where every action that has been 
taken is written down with the current date and time. This log 
also offers the possibility to include comments, thoughts, etc.
with the machine generated ones.
The design goal for the {\bf log} had been that it can
be read easily by the human user.
Beware, as the program generated documenting messages 
are quite extensive even a simple session can produce large output - 
this can be a real nuisance but can also help in some cases!

All the other parts of \period are attached as modules to this backbone.

The most essential module is the one for handling the {\bf time string data}.
It offers the possibility to select different parts of the time string.
Thus each and every data point can have up to 4 different 
{\bf attributes} that can  be attached to it. 
With these attributes, which are named by default:
{\it Date}, {\it Observatory}, {\it Observer} and {\it Other},
each data point can have a label of its own. With this it is easy 
for example to differentiate between days, 
observers or observatories - even techniques
or different instrumentation.

An example for a data point is:

\begin{tabular}{|c|c|}\hline
\row{{\bf time}&{116.7018519}}
\row{{\bf amplitude}&{11.231}}
\row{{\bf date}&{1996-02-03}}
\row{{\bf observatory}&{KittPeak}}
\ruledrow{{\bf observer}&{Smith}}
\end{tabular}

and the format of the data would be:\\
{\tt 116.7018519     11.231  1996-02-03      KittPeak    Smith}

Thus selecting a certain subset of the time string can be done 
by selecting the appropriate labels in each of these attributes. 

\note{A point is only added to the current selection if 
{\bf all} its attributes are selected.}

These qualities also offer common additional attributes except for 
their name, namely:
\myitemize{
\item {\bf color}: this helps to differentiate between different 
        attributes in a graph.
\item {\bf weight}: this gives, as a product of the weight of all
        qualities for a certain point, a weight to the point. 
        This value then can be used by all the other modules
        when performing calculations.
}

Finally there is the concept of {\bf uncorruptable data},
which means, that the original data should not be shifted, scaled,
etc. in any way. So there are two distinct sets of amplitudes for each 
data point:
\myitemize{
\item {\bf original}: the unedited data
\item {\bf adjusted}: the edited data, which can easily be reverted
        to the original values.
}

\note{Again, as \period never wants to change original data, the deletion of 
data is not an act of deleting but an act of changing a certain
attribute to ``deleted'' for example. (This label can be changed.)}

The next module is only responsible for {\bf sinusoidal fits} to the
currently selected time string. 
It basically consists of a table of frequencies,
amplitudes and phases along with a zero point.

The fit to the data is as follows:\\
\latexonly{$A(t)=Z+\sum A_{i} sin(2\pi(t F_{i}+P_{i})) $}
\latexignore{$A(t)=Z+SUM(A{i} sin(2 \pi(t F{i}+P{i})))$\\}

Frequencies can be active or inactive for a certain fit,
giving more degrees of freedom to the user. The user may check different
solutions to the same time string, or look for effects of certain frequencies
on a Fourier spectrum.

Additional features are that a frequency can either be a {\bf harmonic} 
of another frequency (which helps fitting data with nonsinusoidal
shapes) or a {\bf combination} of two other frequencies 
(sum or difference of integer multiples of each frequency).

Fits to the data are allowed with almost arbitrary freedom,
but basically there are 3 modes of calculation:
\myitemize{
\item {\bf Calculate}: keeps the frequencies fixed and
        finds the best fit for all selected amplitudes and phases.
\item {\bf Improve}: improves the frequencies as well.
\item {\bf Special}: gives full choice of what frequencies,
        amplitudes and phases should be improved, as well as
        a possibility to calculate 
        {\bf amplitude and/or phase variations }
        according to a certain attribute of the time string.
}

The last of the modules of \period gives access to {\bf Fourier} techniques.
This has some interconnections with the {\bf fit} module. 

\note{it performs full featured Fourier transformation and does not use
the faster ``Fast Fourier Transform ''(FFT) algorithm, as data
do not have to be equally spaced.}

For bookkeeping, the result of the Fourier calculations are stored
together with the project as long as these are not removed by the user.
Each calculation also has a label of its own for ease of recognition.

One special feature is the possibility to calculate
a Fourier spectrum with {\it weighted} data, 
or the possibility to calculate 
{\bf Fourier noise}. This means calculating an average amplitude for 
a certain frequency range. It can be very useful to determine if 
an peak is significant or not.

%%%%%%%%%%%%%%%%%%%%%%%%%%%%%%%%%%%%%%%%%%%%%%%%%%%%%%%%%%%%%%%%%%%%%%%%%%%%%%%
\section{Starting up \periodname}
\label{basic.start}

Starting \period is as easy as: 
\myitemize{
\item typing {\tt \periodname} under Unix(tm)
\item Starting \period from the start menu under
         Windows 95\registered , or
         Windows NT\registered
}

%%%%%%%%%%%%%%%%%%%%%%%%%%%%%%%%%%%%%%%%%%%%%%%%%%%%%%%%%%%%%%%%%%%%%%%%%%%%%%%
\section{A small example}
\label{basic.example}

%%%%%%%%%%%%%%%%%%%%%%%%%%%%%%%%%%%%%%%%%%%%%%%%%%%%%%%%%%%%%%%%%%%%%%%%%%%%%%%
\subsection{Reading in a time string}
First we have to load a time string.
To do so, please go to the 
\helpref{time string folder}{timestring.folder} and press
the \button{Import} in the bottom of the window.
Then you will be asked for the name of the file you want to import.
Please select the file {\tt test.dat} in the samples directory of
the \period installation directory 
(the exact location depends on your installation, but it should be
installed in the directory ``samples'' of the installation directory 
of \period).

Then you will see that the \helpref{import dialog window}{timestring.import} 
will show up. For the moment please just click on the \button{Ok},
as the defaults are acceptable.
For further information please check out the reference above.

As you can see, the window has changed and some further information
is given, e.g. the number of points selected and the total number of
points.

To have a look at the data please press on the \button{Display Graph}.
This will open up a window with the representation of the timestring.
(See \helpref{timestring graph window}{timestring.graph} for more details.)

You already can see some periodicity in the graph itself, so 
let us try to find out what frequencies are in the data.

%%%%%%%%%%%%%%%%%%%%%%%%%%%%%%%%%%%%%%%%%%%%%%%%%%%%%%%%%%%%%%%%%%%%%%%%%%%%%%%
\subsection{Fourier analysis}
First we have to make a Fourier analysis.
Activate the \helpref{fourier folder}{fourier.folder} by clicking
with the mouse on {\bf Fourier} - but do {\it not} select the menu.

The new dialog offers a lot of possibilities, but as the defaults
for \period are well chosen in this case just pressing on the
\button{Calculate} will start the calculation.

But before starting the calculation \period will ask if it should
subtract an average zero point of -0.007892. 
(For more detailed discussion please see 
\helpref{Fourier zero point}{fourier.zeropoint}.)
Please choose yes as well and then the calculations will start.

When the calculation has finished, a small window will show up and ask, 
if you want to include the frequency with the highest peak. 
(The result should be: 
Frequency of 11.9162641 
and an amplitude of 0.10172852).
Please choose again yes and this frequency will be included 
in a list of frequencies.

The Fourier spectrum can also be examined graphically by pressing on the
\button{Display graph}.
The new \helpref{window opening up}{fourier.graph} will display the result.

\note{As you may notice the graph is not smooth.
This is due to the fact that during Fourier calculation 
\period tries to minimize
memory usage by memorising only local extrema and skipping all the
points in between.}

%%%%%%%%%%%%%%%%%%%%%%%%%%%%%%%%%%%%%%%%%%%%%%%%%%%%%%%%%%%%%%%%%%%%%%%%%%%%%%%
\subsection{Fit one frequency}
So, as we have got a frequency, we can try to fit it to the signal.
Activate the \helpref{fit folder}{period.folder} by clicking
with the Mouse on {\bf Fit} - but not on the menu.

As we have already included one frequency, 
we can see its parameters in the table as F1.
As the frequency F1 has not been activated,
please tick the checkbox in front of {\bf F1} to activate it.

Now, at first we have to find out about the true phase
using the least square fit technique.
Just press on the \button{Calculate} and almost immediately
the window will be updated with the new values.

\begin{tabular}{|c|c|c|}\hline
\row{{\bf frequency}&{\bf amplitude}&{\bf phase}}\hline
\row{{11.9162641}&{0.100150376}&{0.0200245}}
\hline\hline
\row{{\bf zero point}&{0.00101803983}&}
\ruledrow{{\bf residuals}&{0.0342341403}&}
\end{tabular}

But the frequency itself has not been improved. 
And this is what we actually want to do.
So press now on the \button{Improve all}.
This will also improve the frequency and should give the
following result:

\begin{tabular}{|c|c|c|}\hline
\row{{\bf frequency}&{\bf amplitude}&{\bf phase}}\hline
\row{{11.9870449}&{0.101277217}&{0.00447894}}
\hline\hline
\row{{\bf zero point}&{0.00224986037}&{}}
\ruledrow{{\bf residuals}&{0.0336615961}&{}}
\end{tabular}

This is the best fit that we can reach with a single frequency.

%%%%%%%%%%%%%%%%%%%%%%%%%%%%%%%%%%%%%%%%%%%%%%%%%%%%%%%%%%%%%%%%%%%%%%%%%%%%%%%
\subsection{Fourier analysis with residuals}
Now go back to the {\bf Fourier folder} and
select the ``Residuals at original'' entry in the 
``Calculation based on'' item.
Then calculate Fourier anew by pressing the \button{Calculate}.

This time you should not be prompted for a zero point, as \period
assumes that the residuals have a mean value of zero. 
The fit should already have taken care of it.

Again please include the frequency of the highest point, which in this case
should be: Frequency= 7.16586151 and amplitude: 0.0457409859.

%%%%%%%%%%%%%%%%%%%%%%%%%%%%%%%%%%%%%%%%%%%%%%%%%%%%%%%%%%%%%%%%%%%%%%%%%%%%%%%
\subsection{Fit two frequencies}

Now we have found two frequencies, so let's go back to the 
{\bf Fit folder} and activate the frequency F2 as well.
Then press again the \button{Calculate}.

After the calculations have finished,
the window will be updated with the new values:

\begin{tabular}{|c|c|c|}\hline
\row{{\bf frequency}&{\bf amplitude}&{\bf phase}}\hline
\row{{11.9870449}&{0.0975994208}&{0.00428761}}
\row{{7.16586151}&{0.0479167213}&{0.25244}}
\hline\hline
\row{{\bf zero point}&{-0.00198521031}&{}}
\ruledrow{{\bf residuals}&{0.00645800843}&{}}
\end{tabular}

Finally improve the frequencies as well
by pressing on the \button{Improve all},
and the window will be updated with
the new values:

\begin{tabular}{|c|c|c|}\hline
\row{{\bf frequency}&{\bf amplitude}&{\bf phase}}\hline
\row{{11.9994309}&{0.100072311}&{0.000117863}}
\row{{6.99670604}&{0.0501719796}&{0.301063}}
\hline\hline
\row{{\bf zero point}&{1.50619639e-05}&{}}
\ruledrow{{\bf residuals}&{0.000281511948}&{}}
\end{tabular}

This is very close to the values chosen for creating
the artificial light curve, which are:

\begin{tabular}{|c|c|c|}\hline
\row{{\bf frequency}&{\bf amplitude}&{\bf phase}}\hline
\row{{12}&{0.1}&{0.0}}
\row{{7}&{0.05}&{0.3}}
\hline\hline
\ruledrow{{\bf zero point}&{0.0}&{}}
\ruledrow{{\bf residuals}&{0.000329267906}&{artificial noise}}
\end{tabular}

%%%%%%%%%%%%%%%%%%%%%%%%%%%%%%%%%%%%%%%%%%%%%%%%%%%%%%%%%%%%%%%%%%%%%%%%%%%%%%%
\subsection{Save the Project}

Finally we may want to save the full results.
To do so, please select the \menuentry{Save Project} from the 
\menu{File}, and save the project.

\period can be quit by selecting the \menuentry{Quit} from the 
\menu{File}.

To continue with all the settings that you have had in your last session
restart \period and then select the \menuentry{Load Project} 
from the \menu{File}. Now you can continue as if the work has
not been interrupted.

%%% Local Variables: 
%%% mode: latex
%%% TeX-master: "period98"
%%% End: 
