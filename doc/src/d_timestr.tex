\section{Time string}%
\label{timestring.detailed}

%%%%%%%%%%%%%%%%%%%%%%%%%%%%%%%%%%%%%%%%%%%%%%%%%%%%%%%%%%%%%%%%%%%%%%%%%%%%%%%
\subsection{Time string folder}%
\label{timestring.folder}
\begin{figure}[hb]
$$\image{0cm;0cm}{TFolder.eps}$$%
\caption{The ``time string'' folder}%
\label{timestring.folder.dialog}
\end{figure}

This folder shows all the required information regarding the time string.

{\bf Current data file}: 
gives the full pathname 
of the files from which the data have been imported.

{\bf Points selected}: 
gives the number of points that are selected.

{\bf Total points}: 
gives the total number of points in the time string.

{\bf Start time}:
gives the start time of the selected timestring.

{\bf Time string is in magnitudes}:
informs \periodname that the timestring is in magnitudes 
(i.e. the scale is inverted for plots, etc\ldots)

{\bf End time}:
gives the end time of the selected timestring.

There are {\bf 4 list boxes with buttons on top} side by side,
each of which represents one attribute of the time string.

Please note that the identifiers for each attribute can be changed
to fit the requirements by pressing on the button on top of each list box.

{\bf Edit Buttons below the list boxes:}
these allow changes to some settings in the currently 
highlighted labels of an attribute - i.e: this allows changes to
the weight and color of this label.
See also \helpref{Edit label}{timestring.edit}.

The 4 Buttons in the bottom have the following meaning:
\myitemize{
\item{\bf Display data}: 
Displays the selected time string with the full information regarding 
residuals, attribute, weights, etc. in a 
\helpref{window}{timestring.table}.
\item{\bf Import}:
Imports a time string from file.
See also \helpref{Import Time string}{timestring.import}
\item{\bf Export}:
Exports the selected time string to a file.
See also \helpref{Export Time string}{timestring.export}
\item{\bf Display Graph}:
Displays the selected time string in a 
\helpref{window}{timestring.graph} with the current fit.
}

%%%%%%%%%%%%%%%%%%%%%%%%%%%%%%%%%%%%%%%%%%%%%%%%%%%%%%%%%%%%%%%%%%%%%%%%%%%%%%%
\subsection{Import time string}%
\label{timestring.import}

There are the following possibilities to import a time string:
\myitemize{
\item the \button{Import} is pressed in the time string folder
\item the \menuentry{Import time string} in the 
        \menu{time string} has been selected
\item the \menuentry{Append time string} in the \menu{time string}
        has been selected. In this case the old time string does not
        get erased but the next time string is appended to the current
        one.
}

A common file selector dialog will open up to select a file to import.
Then the file is parsed for its contents and finally the 
\helpfigref{Import format dialog}{timestring.import.dialog} is opened.
\begin{figure}[h]
$$\image{0cm;0cm}{TImport.eps}$$%
\caption{The ``import format'' dialog}%
\label{timestring.import.dialog}
\end{figure}

This dialog lists the first few lines  of the time string to read in 
and shows them along with some defaults
for each column, guessed from the structure of the data in the file.
Only 4 columns are displayed at every given time, but scrolling is 
achieved by pressing the buttons to the left and right of the list box.

Heading each column is a choice item with the following possibilities:
\myitemize{
\item Time
\item Observed
\item Adjusted
\item Residuals to Observed
\item Residuals to Adjusted
\item Calculated
\item Per point weight
\item Attribute 1 - depending on the label given
\item Attribute 2 - depending on the label given
\item Attribute 3 - depending on the label given
\item Attribute 4 - depending on the label given
\item ignore
}
These can be changed to reflect the actual data to read in.
Thus it is possible to ignore certain columns or 
to read in strangely formatted files.

\note{Please note that lines starting with ``\#'', ``;'', ``\%'' 
are assumed to be comment lines and are ignored!}

%%%%%%%%%%%%%%%%%%%%%%%%%%%%%%%%%%%%%%%%%%%%%%%%%%%%%%%%%%%%%%%%%%%%%%%%%%%%%%%
\subsection{Export time string}%
\label{timestring.export}

To reexport the {\it selected} time string there are
two possibilities:
\myitemize{
\item the \button{Export} is pressed in the time string folder
\item the \menuentry{Export time string} in the \menu{time string}
        has been selected
}
This time the output file format is specified using a 
multiple selection list in the 
\helpfigref{export format dialog}{timestring.export.dialog}.

\begin{figure}[h]
$$\image{0cm;0cm}{TExport.eps}$$%
\caption{The ``export format'' dialog}%
\label{timestring.export.dialog}
\end{figure}

This way it is possible to extract only certain 
parts of the full information that each point contains.

\note{In this case the sequence is predetermined 
and always top to bottom. After pressing OK, a dialog to save to 
file will be shown.}

After pressing the \button{Ok} a dialog to save to a file will be shown.

\note{As a final note: The time string will be written out in the same 
sequence as it had been read in. So reading in unsorted data
and re-exporting it again will not result in a sorted time string.}

%%%%%%%%%%%%%%%%%%%%%%%%%%%%%%%%%%%%%%%%%%%%%%%%%%%%%%%%%%%%%%%%%%%%%%%%%%%%%%%
\subsection{Combine substrings}%
\label{timestring.rename}

It is possible to combine the selection and to give it a common
attribute. To do so, the \menuentry{Combine substrings} in the 
\menu{time string} has to be selected.

The \helpfigref{Combine substrings dialog}{timestring.rename.dialog},
which opens up gives the possibility to enter a new common label and 
select which attribute the label should belong to.
\begin{figure}[h]
$$\image{0cm;0cm}{TRename.eps}$$%
\caption{The ``combine substrings'' dialog}%
\label{timestring.rename.dialog}
\end{figure}

%%%%%%%%%%%%%%%%%%%%%%%%%%%%%%%%%%%%%%%%%%%%%%%%%%%%%%%%%%%%%%%%%%%%%%%%%%%%%%%
\subsection{Subdivide selection}%
\label{timestring.subdivide}

Often a time string contains gaps due to a specific observational pattern.
In this case, \period offers the possibility to split a currently selected
time string into subsets by finding gaps that fulfill a requirement and 
splitting it up there.

To do so the \menuentry{subdivide time string} in the \menu{time string}
has to be selected. Then the 
\helpfigref{subdivide selection dialog}{timestring.subdivide.dialog}
will be displayed and the relevant data for this operation can be entered:
\myitemize{
\item {\bf minimum size of gap} between 2 consecutive points
        at which to make a subdivision.
\item what {\bf attribute} the separations should belong to.
\item what is the {\bf label prefix} from which to create a final label.
\item should a {\bf running counter} or the integer part
        of the time of its first new element 
        be added to the label prefix to give the resulting label.
}

\begin{figure}[h]
$$\image{0cm;0cm}{TSubdivide.eps}$$%
\caption{The ``subdivide selection'' dialog}%
\label{timestring.subdivide.dialog}
\end{figure}

\note{To reconvert the subdivided timestring, removing all the generated
labels and replacing them with a new common one, the 
\helpref{Combine substrings}{timestring.rename} option should be used.
A new label should be entered and as attribute the attribute
where all the split labels are should be entered.}

%%%%%%%%%%%%%%%%%%%%%%%%%%%%%%%%%%%%%%%%%%%%%%%%%%%%%%%%%%%%%%%%%%%%%%%%%%%%%%%
\subsection{Adjust selection}%
\label{timestring.adjust}
Often different subsets of data can have marginally different 
zero point offsets, which may be due to different instrumentation or
not fully ideal situation for observation. (In astronomy, different
seeing for each night may be an example for this, or telescopes 
at different sites around the world.)

These zero point variations produce noise in the low frequency domain, which
can be reduced with this technique.

To compensate for these effects the \menuentry{Adjust selection} in the 
\menu{time string} has to be selected. Before this, a fit to the data should be
done, as this routine uses the residuals to calculate the residual
zero points for the different labels of an attribute.

The 
\helpfigref{adjust zero point for selection dialog}{timestring.adjust.dialog}
has four buttons on top with which the selection of the attribute 
to be investigated can be chosen.
\begin{figure}[h]
$$\image{0cm;0cm}{TAdjust.eps}$$%
\caption{The ``adjust zero point for selection'' dialog}%
\label{timestring.adjust.dialog}
\end{figure}

The {\bf use weights} check box indicates whether weighted residuals have been 
used to calculate the average and sigma.
The list box lists all the labels of the selected attribute along with the 
average, sigma and number of points for this label. 

If the last column
contains a ``yes'', this means that some or all of the points have already
been adjusted before.

To adjust some labels, they should be selected in the list box and then the 
\button{Ok} should be pressed.
After each adjustment, a new fit should be calculated to reflect the new 
situation of the data. 

The current list can also be printed by pressing the \button{Print} 
or can be saved to file by pressint the \button{Save}.

\note{Please do not forget to select the adjusted data
from this point on for calculations.}

%%%%%%%%%%%%%%%%%%%%%%%%%%%%%%%%%%%%%%%%%%%%%%%%%%%%%%%%%%%%%%%%%%%%%%%%%%%%%%%
\subsection{Edit label}%
\label{timestring.edit}
This  
\helpfigref{edit label dialog}{timestring.edit.dialog}
is shown when one of the \button{Edit}%
s in the 
\helpref{time string folder}{timestring.folder}
has been pressed. It offers the possibility
to change the color and weight
for each selected label in the attribute.
\begin{figure}[h]
$$\image{0cm;0cm}{TEdit.eps}$$%
\caption{The ``edit label'' dialog}%
\label{timestring.edit.dialog}
\end{figure}

%%%%%%%%%%%%%%%%%%%%%%%%%%%%%%%%%%%%%%%%%%%%%%%%%%%%%%%%%%%%%%%%%%%%%%%%%%%%%%%
\subsection{Time string table}%
\label{timestring.table}
When the \button{Display table} in the 
\helpref{time string folder}{timestring.folder}
is pressed, a
\helpfigref{window}{timestring.table.window}
is shown. This displays the selected time string in every detail.
A printout of this table is possible by selecting the \menuentry{Print}
in the \menu{File} of that window.
The attribute labels for a certain point can be edited by clicking with the 
right mouse button in that line. 
(See \helpfigref{Change attribute label}{timestring.changeattribute.dialog})

\begin{figure}[h]
$$\image{0cm;0cm}{TTable.eps}$$%
\caption{The ``time string data'' window}%
\label{timestring.table.window}
\end{figure}

\note{With the {\it control key on the keyboard} pressed 
at the same time as the right mouse button,
the data point is immediately deleted without further questioning.}

%%%%%%%%%%%%%%%%%%%%%%%%%%%%%%%%%%%%%%%%%%%%%%%%%%%%%%%%%%%%%%%%%%%%%%%%%%%%%%%
\subsubsection{Change attribute label for a point}%
\label{timestring.changeattribute}
This allows to change all the labels of a certain point
of the time string by clicking with the right mouse button
on the data point in the Graph of Table view.
This opens up the 
\helpfigref{Change attribute label for a point}{timestring.changeattribute.dialog}
dialog, in which the labels for the selected point can be changed.
\begin{figure}[h]
$$\image{0cm;0cm}{TChangeQ.eps}$$%
\caption{The ``change attribute label for a point'' dialog}%
\label{timestring.changeattribute.dialog}
\end{figure}

\note{After renaming a point, it is not necessarily any
longer in the currently selected time string, 
as the newly given label may not be selected.}

%%%%%%%%%%%%%%%%%%%%%%%%%%%%%%%%%%%%%%%%%%%%%%%%%%%%%%%%%%%%%%%%%%%%%%%%%%%%%%%
\subsection{Time string graph}%
\label{timestring.graph}
When the \button{Display graph} in the 
\helpref{time string folder}{timestring.folder}
is pressed, a
\helpfigref{window}{timestring.graph.window}
is shown. This displays the selected time string as
colored dots together with a curve representing the current fit.
\begin{figure}[h]
$$\image{0cm;0cm}{TGraph.eps}$$%
\caption{The ``time string graph'' window}%
\label{timestring.graph.window}
\end{figure}

There are 4 options in the \menu{Data}, which allow
to select the kind of data to investigate:
\myitemize{
\item Original: displays the original data together with the current fit
\item Adjusted: displays the adjusted data together with the current fit
\item Residuals(obs): displays the residuals to the observed values.
\item Residuals(adj): displays the residuals to the adjusted values.
}

\note{The current fit is {\it only} plotted, 
when the resolution of the screen is high enough to display the fit
without any sampling problems.}

A printout of this graph is possible by selecting the \menuentry{Print}
in the \menu{File} of that window.

Similar to the 
\helpref{Time string table window}{timestring.table}, points can be renamed
or deleted. 

\note{Beware: The cursor has to be close to the point itself
and no other point may be close. If these conditions are not
fulfilled, then no point is deleted.
A beep signifies that a point has been deleted.}

The colors are defined by the color of the labels of the attribute that is
selected in the \menu{Color}.

The graph also allows zooming into the data.
Selecting an area with the left mouse button zooms into that
area. 
Changing the displayed area can be achieved by the
\menuentry{Select view port} in the \menu{Zoom}. See
\helpref{Change view port}{timestring.graph.viewport} 
for exact explanations.
Finally the original view can be restored by selecting
\menuentry{Display all} in the \menu{Zoom}.

The status bar writes the current position of the mouse, while it is over the
display.

%%%%%%%%%%%%%%%%%%%%%%%%%%%%%%%%%%%%%%%%%%%%%%%%%%%%%%%%%%%%%%%%%%%%%%%%%%%%%%%
\subsubsection{Change view port}%
\label{timestring.graph.viewport}
To invoke the 
\helpfigref{Change view port dialog}{timestring.graph.viewport.dialog},
the \menuentry{Select view port} in the \menu{Zoom} has to be selected 
in the
\helpref{Time string graph}{timestring.graph}.
There the current values of the view port are listed and can be changed.
\begin{figure}[h]
$$\image{0cm;0cm}{GGraphViewport.eps}$$%
\caption{The ``change view port'' dialog}%
\label{timestring.graph.viewport.dialog}
\end{figure}

%%%%%%%%%%%%%%%%%%%%%%%%%%%%%%%%%%%%%%%%%%%%%%%%%%%%%%%%%%%%%%%%%%%%%%%%%%%%%%%
\subsection{Delete default label}%
\label{timestring.default}
To change the default values of attribute and label 
for deleted points, the \menuentry{Delete default label} in
the \menu{time string} of the main window has to be selected.

This opens up the 
\helpfigref{delete default label dialog}{timestring.default.dialog}.
\begin{figure}[h]
$$\image{0cm;0cm}{TDefault.eps}$$%
\caption{The ``delete default label'' dialog}%
\label{timestring.default.dialog}
\end{figure}

This is only relevant in the time string 
\helpref{table}{timestring.table} and 
\helpref{graph}{timestring.graph}.

%%% Local Variables: 
%%% mode: latex
%%% TeX-master: "period98"
%%% End: 
