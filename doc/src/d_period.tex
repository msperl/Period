\section{Fit}%
\label{period.detailed}

%%%%%%%%%%%%%%%%%%%%%%%%%%%%%%%%%%%%%%%%%%%%%%%%%%%%%%%%%%%%%%%%%%%%%%%%%%%%%%%
\subsection{Fit folder}%
\label{period.folder}

\begin{figure}[h]
$$\image{0cm;0cm}{PFolder.eps}$$%
\caption{The frequency folder}%
\label{period.folder.dialog}
\end{figure}

This folder shows almost all the information regarding the current fit. 
(In some rare circumstances the 
\helpref{log folder}{log.detailed}
may provide more detailed information.)

{\bf Active frequencies}:
Gives the total number of active frequencies.
These are the frequencies for which in the frequency table,
the check box is selected near the frequency number.

{\bf Use weights}
means: use the time string in a weighted form for the
least square fit calculation, if it has been selected.

{\bf Zero point}:
Shows the calculated zero point for the last calculated fit.

{\bf Residuals}:
Shows the residuals for the last calculated fit.

{\bf Calculations based on:} dialog item indicates which data set
will be used for the next calculation. Either the {\bf original} 
or the {\bf adjusted} data.

With the four buttons:
{\bf Prev X},
{\bf Prev},
{\bf Next},
{\bf Next X}
scrolling in the frequency list is possible.
(Prev is short for previous.)
(X depends on the size of the screen, 
but can also be specified with a 
command line switch of {\tt -r$ <$rows$> $} when invoking the 
program.)

The next view lines give information on some of the frequencies themselves.

The first column gives the {\bf number of the frequency}
with the possibility to activate or deactivate it by
selecting the check box.

The next column gives the value of the {\bf frequency} itself.
In case of a {\bf harmonic} or {\bf frequency combination}
the contents will be preceded by {\tt``=''} and may be 
of one of the following formats:

frequency combination:
\myitemize{
        \item =fx+fy
        \item =nfx+mfy or =n*fx+m*fy
}

harmonic:
\myitemize{
        \item =hfx or =h*fx
}

x,y may be values from 1 to 255, but not be the number
of the frequency itself or a reference to another frequency
which is itself a combination. If one of the composing 
frequencies is not active, the combination is not active either! 
h may be an integer greater than 1, while m,n can be any integer number.

When entering or editing a frequency value, another feature may make life easy:
if an arbitrary number of ``+'' and ``-'' are trailing the number, then
the value of the 
\helpref{alias gap value}{period.aliasgap} is added to or subtracted from
the value, as often as ``+'' or ``-'' are given.

The following two columns give the value of the {\bf amplitude}
and {\bf phase}.

The last column, which is by default not visible, tells
the current numerical value of a composite frequency or harmonic.

The next 3 buttons start a calculation:
\myitemize{
\item \button{Calculate}
takes the current settings and improves 
the amplitudes and phases of all active frequencies.

\item \button{Improve all}
takes the current settings and improves 
not only amplitudes and phases, but also the frequencies.

\item \button{Improve Special}
gives full power to the user for calculations, as it allows to 
decide for every active frequency, amplitude or phase whether it 
should be a fixed or free parameter.
This option also allows the calculation of 
\helpref {amplitude and/or phase variations}{period.ampvar}.
}

{\bf Import}
reads in a set of predefined frequencies.

{\bf Export}
writes out the whole table of frequencies.

{\bf Phase plot}
allows to display the selected time string as a phase plot to
a selectable frequency.

%%%%%%%%%%%%%%%%%%%%%%%%%%%%%%%%%%%%%%%%%%%%%%%%%%%%%%%%%%%%%%%%%%%%%%%%%%%%%%%
\subsection{Alias gap}%
\label{period.aliasgap}
As already mentioned in the
\helpref{Fit folder}{period.folder},
a frequency may be shifted up or down by some arbitrary value by placing
``+'' or ``-'', after the frequency value itself.
This value is closely connected to the
spectral window function of the currently selected time string.
For astronomical reasons, the default is set to 1/365. 
This value is a good estimate for a time string
with large gaps (of approximately a year) between 
densely packed data.

The \helpfigref{alias gap dialog}{period.aliasgap.dialog},
which can be reached from the \menu{Fit} by selecting the 
\menuentry{Alias gap}, allows to change this value.
\begin{figure}[h]
$$\image{0cm;0cm}{PAliasgap.eps}$$%
\caption{The ``alias gap'' dialog}%
\label{period.aliasgap.dialog}
\end{figure}

%%%%%%%%%%%%%%%%%%%%%%%%%%%%%%%%%%%%%%%%%%%%%%%%%%%%%%%%%%%%%%%%%%%%%%%%%%%%%%%
\subsection{Improve Special}%
\label{period.special}
As already mentioned in the description of the 
\helpref{Fit folder}{period.folder}, {\bf Improve special} gives
full control over the calculation to make. 

\note{This option should only be used with great care 
and only after a stable solution has already been reached with the 
\button{Calculate} or \button{Improve}.}

\note{Improve automatically 
does a calculate before trying to improve the frequencies as well! 
This safeguard has not been implemented for {\bf Improve Special},
as this might effect the desired result in unwanted ways.}

The \helpfigref{improve special dialog}{period.special.dialog}
basically contains 3 list boxes. 
These list the frequency, amplitude, and phase for every active frequency.
For ease of recognition, in front of each value the number of the 
frequency itself is written.
As for the frequency list, frequencies that are combinations or 
harmonics are not listed.

\begin{figure}[h]
$$\image{0cm;0cm}{PSpecial.eps}$$%
\caption{The ``improve special'' dialog}%
\label{period.special.dialog}
\end{figure}

As an example, a list of predefined frequencies (f1, f2 and f3) is already 
known, and should stay fixed. But other additional frequencies (f4 to f10) 
should be improved. 
Then in the list box with the frequencies all frequencies
but the first three should be selected, as well as all the amplitudes
and phases.

The three buttons in the bottom allow different calculation modes.
\myitemize{
\item \button{Calculate} gives the requested calculation
\item \button{Calculate amplitude/phase variations} opens up another dialog,
        giving access to an even more specialised calculation mode.
        See \helpref{Amplitude/phase variations}{period.ampvar}
        for details.
\item \button{Cancel} stops the calculation.
}

\note{Interpretation of the log output of the {\bf Fit} module:\\
It will output a before/after scenario with all the active Frequencies listed.
Values that have leading and trailing asterisk are free parameters to the fit.
The others are fixed. }

\note{And a final note: the {\bf zero point} is {\it always} a free parameter!}

%%%%%%%%%%%%%%%%%%%%%%%%%%%%%%%%%%%%%%%%%%%%%%%%%%%%%%%%%%%%%%%%%%%%%%%%%%%%%%%
\subsection{Calculate amplitude/phase variations}%
\label{period.ampvar}

The last and most sophisticated calculation mode provided by the
{\bf fit module} is the {\bf Amplitude/Phase variation mode}.
This allows to analyze data where 
amplitudes and/or phases have different values, for different subsets 
of the currently selected time string.

An example for this is the observation of daily 
high and low temperature averages.
Or a star that changes intrinsically its amplitude for some frequency. 
Also measuring an object in two filters generally gives different amplitudes
and phases for each filter.

Why is this calculation mode necessary? 
Fitting different subsets is possible 
and should give the same results.

This is not fully true, as precision of the final result
gets lost. Not only because of the limited number of points, but also
because the frequency for each of these subsets will yield a slightly 
different result when using {\bf improve} for calculation. 
Thus the results are not fully comparable, especially when large datasets
with huge gaps (say maybe 10 years) are used.

But will {\bf calculate} or {\bf improve special} not do the magic to 
some extent? True again, but again a certain frequency has to be assumed
from the start for all subsets, which will still not represent the
best fit.

But a blind eye shot with this new tool will not give necessarily the
wanted results either. Because, when the degrees of freedom are increased
for the calculations (and that is what amplitude/phase variation does!),
the numerical stability may be compromised.

So this tool should only be used with great caution and the 
number of frequencies,
which are tested for amplitude and/or phase variations, should be kept 
to a minimum. Other techniques should be used first to find such variations
and only {\it then} this tool should be used.

The other techniques are: calculating values for different subsets
and the more optical approach of using the
\helpref{phase plot}{period.phase} feature of the {\bf fit module}.

After the \button{Calculate amplitude/phase variations} has been pressed
in the \helpref{Improve special dialog}{period.special}, the
\helpfigref{amplitude/phase variation dialog}{period.ampvar.dialog} shows up.

\begin{figure}[h]
$$\image{0cm;0cm}{PAmpvar.eps}$$%
\caption{The ``amplitude/phase variation'' dialog}%
\label{period.ampvar.dialog}
\end{figure}

First an {\bf attribute} has to be chosen, for which its selected labels 
define subsets of the time string.

Then the {\bf calculation mode} has to be selected. 
Possibilities are:
\myitemize{
\item amplitude variation
\item phase variation
\item amplitude and phase variation
}

And finally the {\bf frequencies} for which amplitude/phase 
variation should be done, have to be selected. The ones not selected 
are assumed to have the same amplitude and phase for all points.

The \button{Calculate} starts the calculation. 
As the \helpref{fit folder}{period.folder} does not offer
the possibility to show the complex results in its window,
a new dialog shows up, which tabulates the result.

The \helpref{fit folder}{period.folder} will only reflect 
changes to the values, for which separate amplitude and phase variation
has not been done.

The residuals are calculated correctly and individual residuals for
the subsets can be examined with the 
\helpref{Adjust selection}{timestring.adjust} tool in the
\helpref{Time string module}{timestring.folder}.

%%%%%%%%%%%%%%%%%%%%%%%%%%%%%%%%%%%%%%%%%%%%%%%%%%%%%%%%%%%%%%%%%%%%%%%%%%%%%%%
\subsection{Calculate message}%
\label{period.calculate}
When calculating a fit the main window will change its appearance 
to the following:
\begin{figure}[h]
$$\image{0cm;0cm}{PCalculate.eps}$$%
\caption{The ``calculate status'' window}%
\label{period.calculate.message}
\end{figure}

It informs about the current state of calculations, 
where the first number is the number of iterations already done.
The second number tells at what number of iterations the program will stop 
iterating and ask for advice.

With the \button{Cancel} the calculation can be stopped prematurely.
In this case, the result may not be a stable solution.

%%%%%%%%%%%%%%%%%%%%%%%%%%%%%%%%%%%%%%%%%%%%%%%%%%%%%%%%%%%%%%%%%%%%%%%%%%%%%%%
\subsection{Predict signal}%
\label{period.predict}
With the \menuentry{Predict} in the \menu{Fit} it is possible to 
predict an amplitude at a certain time from the current fit.
The \helpfigref{predict amplitude dialog}{period.predict.dialog} only
allows to enter a {\bf time} and the \button{Calculate}
will update the display accordingly.
\begin{figure}[h]
$$\image{0cm;0cm}{PPredict.eps}$$%
\caption{The ``predict amplitude'' dialog}%
\label{period.predict.dialog}
\end{figure}

%%%%%%%%%%%%%%%%%%%%%%%%%%%%%%%%%%%%%%%%%%%%%%%%%%%%%%%%%%%%%%%%%%%%%%%%%%%%%%%
\subsection{Create artificial data}%
\label{period.artificial}
With the
\helpfigref{create artificial data dialog}{period.artificial.dialog},
which can be activated by selecting the
\menuentry {Create artificial data} from the \menu{Fit}, 
it is possible to create an equally spaced artificial time string.
\begin{figure}[h]
$$\image{0cm;0cm}{PArtificial.eps}$$%
\caption{The ``create artificial data'' dialog}%
\label{period.artificial.dialog}
\end{figure}

To do so, \period has to know about the {\bf start} and {\bf end time} 
for the selected time span
and as well the {\bf steps} in between
defaults for this values are: first and last time of the 
currently selected time string and for the step 1/(20*Fmax) to give a good 
sampling for the highest frequency as well.

{\bf Leading/trailing} allows to extend the time span defined by start
and end in both directions further by a common value.

After pressing the \button{Append to file} or the \button{Create new file}
a file needs to be selected. For {\it append} the data will be appended
to the file, and with {\it create} a new file will be created, and 
the old file destroyed.


To create artificial data with times from the currently selected time string,
the ``wished'' frequencies should be entered in the 
\helpref{Fit folder}{period.folder} and then the 
\menuentry{Recalculate residuals} in the \menu{Fit} should be used to 
\helpref{recalculate the residuals}{period.recalculate}.
Then the calculated values can be exported with 
\helpref{Time string export}{timestring.export}.

%%%%%%%%%%%%%%%%%%%%%%%%%%%%%%%%%%%%%%%%%%%%%%%%%%%%%%%%%%%%%%%%%%%%%%%%%%%%%%%
\subsection{Recalculate residuals}%
\label{period.recalculate}
With the \menuentry{Recalculate residuals} in the \menu{Fit}, it is 
possible to update the residuals using the values of the current fit.
In addition, the zero point will be asked in the
\helpfigref{zero point dialog}{period.residuals.dialog}
before the calculations are done.
\begin{figure}[h]
$$\image{0cm;0cm}{PZeropoint.eps}$$%
\caption{The ``zero point'' dialog}%
\label{period.residuals.dialog}
\end{figure}

%%%%%%%%%%%%%%%%%%%%%%%%%%%%%%%%%%%%%%%%%%%%%%%%%%%%%%%%%%%%%%%%%%%%%%%%%%%%%%%
\subsection{Epochs}%
\label{period.epochs}
The time of maximum light closest to a certain time (epoch) according 
to the current fit can be calculated by selecting the
\menuentry{Epoch} in the \menu{Fit}.

Then the \helpfigref{epoch dialog}{period.epochs.dialog} will show up,
which basically lists the table of frequencies along with the epoch.
{\bf Time of epoch} gives the time, for which the epochs should be calculated.
And {\bf Data is in intensity} allows for correct interpretation of 
maximum light. By default magnitudes are assumed!

\begin{figure}[h]
$$\image{0cm;0cm}{PEpochs.eps}$$%
\caption{The ``epoch'' dialog}%
\label{period.epochs.dialog}
\end{figure}

The \button{Calculate} will update these results.

%%%%%%%%%%%%%%%%%%%%%%%%%%%%%%%%%%%%%%%%%%%%%%%%%%%%%%%%%%%%%%%%%%%%%%%%%%%%%%%
\subsection{Import frequencies}%
\label{period.import}
To import a table of frequencies, all that is needed to do is
to press the \button{Import} in the \helpref{Fit folder}{period.folder}%
. Then a file selector dialog will show up and a file to 
read in may be selected.

The table of frequencies will be erased and the information will be filled
in.

The file format is the following for {\it each} line:
\myitemize{
\item Frequency identifier: in the format: F$<$num$>$ or f$<$num$>$
\item if a bracket is the next character, then the frequency is 
        assumed to be inactive
\item frequency
\item amplitude (optional)
\item phase (optional)
\item additional data ignored
}

Lines starting with ``/'', ``;'', ``\#'', ''\%''
are assumed to be comment lines.

%%%%%%%%%%%%%%%%%%%%%%%%%%%%%%%%%%%%%%%%%%%%%%%%%%%%%%%%%%%%%%%%%%%%%%%%%%%%%%%
\subsection{Export frequencies}%
\label{period.export}
To export the table of frequencies, all that is needed to do is
to press the \button{Export} in the \helpref{Fit folder}{period.folder}.
Then a file selector dialog will show up and a filename to save
may be selected.

For the format of the output file please see
\helpref{Import frequencies}{period.import}.

%%%%%%%%%%%%%%%%%%%%%%%%%%%%%%%%%%%%%%%%%%%%%%%%%%%%%%%%%%%%%%%%%%%%%%%%%%%%%%%
\subsection{Phase plot}%
\label{period.phase}
As a last tool, the {\bf Fit module} offers the possibility of phase plotting
the selected time string. This tool has been in use for a long time 
and has had its prime time in the early days of asteroseismology. 
Nowadays this tool is scarcely used, even though
it can prove to be a powerful visual diagnostic tool.
It allows to view the shape of a light curve not in numbers 
but visually. And this possibility can help finding a solution to some,
otherwise possibly overlooked, properties in the data.

The plot basically uses the the reminder of time of a point multiplied by a
certain frequency, which can be edited by the user, 
as abscissa, and as Y-Axis most frequently residuals
(of any kind - Original or Adjusted) to plot a point.

By default, \period tries to find a frequency in the list of frequencies, 
that has not been activated and chooses the first one found.
This frequency is displayed in the top part of the graph. 
If none is found, then a frequency of 1.0 is chosen.

The point itself is plotted as a filled circles with a color that
corresponds to the color of one of its attributes labels.
To change the the attribute to use for to select the color,
the \menu{Color} is used. 
(See \helpref{Edit name properties}{timestring.edit} in the
\helpref{Time string folder}{timestring.folder}) for how to
change the color for a label.)

With the \menu{Data} the data to be displayed on the Y-axis can be selected.

For explanation of the {\bf Color, Data and Zoom menus} please see
the documentation on \helpref{Time string graph}{timestring.graph}.

Some of the possible uses of this graph are:
\myitemize{
\item finding systematic changes of amplitudes 
        in different subsets of the time string.
\item finding systematic changes of phases 
        in different subsets of the time string.
\item detecting ``bad'' data, which do not follow the average
        ``light curve shape'' for whatever reason.
\item reveals the ``true'' ``light curve shape'', as the light curve
        does not have to be sinusoidal, but still periodic.
}

\note{Note: with {\bf harmonic frequencies}, that can be entered
in the frequency table, these light shapes may be ``approximated'' 
quite accurately and  thus be removed from the residuals.} 


So, how does it work? Bringing up the 
\helpfigref{Phase plot}{period.phase.window}
is as easy as pressing the \button{Phase plot} in the
\helpref{Fit folder}{period.folder}.
\begin{figure}[h]
$$\image{0cm;0cm}{PPhase.eps}$$%
\caption{The ``phase plot'' window}%
\label{period.phase.window}
\end{figure}

The \menu{Frequency} gives all the necessary possibilities that have to do
with the graph, like changing the frequency used for calculating
the phase of a point,
or to activate the {\bf binning feature}, which allows to extract the 
light shape ``corresponding''to this frequency. 
For closer detail please see \helpref{Binning}{period.phase.binning}.

The values from which the graph has been created, can be written out in a 
similar manner as the normal 
\helpref{time string export}{timestring.export}%
. The only difference is that instead of time, phase is written out.

%%%%%%%%%%%%%%%%%%%%%%%%%%%%%%%%%%%%%%%%%%%%%%%%%%%%%%%%%%%%%%%%%%%%%%%%%%%%%%%
\subsubsection{Binning}%
\label{period.phase.binning}
As already mentioned, the ``Phase plot'' also offers the possibility of 
binning, which is averaging data for discrete phase ranges 
and displaying these values along with error bars.
Additionally, a spline like curve is fitted to the data resulting from this 
procedure.

The size of the bin box can be changed by selecting the
\menuentry{binning spacing} in the \menu{Frequency}.
Then the \helpfigref{Binning spacing dialog}{period.phase.binning.dialog}
will open up and ask for a new value. This value should be in the 
range of {\bf 0} and {\bf 1} exclusively!
\begin{figure}[h]
$$\image{0cm;0cm}{PPhaseBinning.eps}$$%
\caption{The ``binning'' dialog}%
\label{period.phase.binning.dialog}%
\end{figure}

The binned values can also be exported to a file by selecting the
\menuentry{Export binned} in the \menu{File}. This file contains
3 columns: phase, mean amplitude and sigma of mean.

%%%%%%%%%%%%%%%%%%%%%%%%%%%%%%%%%%%%%%%%%%%%%%%%%%%%%%%%%%%%%%%%%%%%%%%%%%%%%%%
\subsubsection{Frequency choice}%
\label{period.phase.frequency}
If the frequency used in the graph needs to be changed, then the 
\menuentry{Change Frequency} in the \menu{Frequency} should be selected.
This opens up the 
\helpfigref{frequency choice dialog}{period.phase.frequency.dialog}.
\begin{figure}[h]
$$\image{0cm;0cm}{PPhaseFrequency.eps}$$%
\caption{The ``frequency choice'' dialog}%
\label{period.phase.frequency.dialog}
\end{figure}

This Dialogs contains a list box, that
lists all deactivated frequencies, as \period assumes
that activated frequencies already have been prewhitened.
Selecting one of these and pressing the \button{OK} selects
this frequency and updates the graph and binning accordingly.

If a totally different frequency needs to be selected, then there is also
the ``Choose different value'' in the list box.
When this has been selected the 
\helpfigref{custom frequency dialog}{period.phase.frequency.other.dialog}
will show up, and a new frequency can be entered.
\begin{figure}[h]
$$\image{0cm;0cm}{PPhaseFrequencyother.eps}$$%
\caption{The ``custom frequency'' dialog}%
\label{period.phase.frequency.other.dialog}
\end{figure}

%%% Local Variables: 
%%% mode: latex
%%% TeX-master: "period98"
%%% End: 
