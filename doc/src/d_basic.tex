\section{Basics}%
\label{basic.detailed}

\subsection{The main window}

\begin{figure}[h]
$$\image{0cm;0cm}{GMainTop.eps}$$%
\caption{The top part of the main window of \period}%
\label{mainframe.top}
\end{figure}
When starting \period is the main window will be shown.
In the \helpfigref{top}{mainframe.top} there is a menu bar and 
below 4 folders labeled as:
\myitemize{
\item \helpref{Time string}{timestring.detailed}, 
\item \helpref{Fit}{period.detailed}, 
\item \helpref{Fourier}{fourier.detailed} and
\item \helpref{Log}{log.detailed}.
}
Each of these can be clicked to enter the relevant folder.

\begin{figure}[h]
$$\image{0cm;0cm}{GMainBottom.eps}$$%
\caption{The lower part of the main window of \period}%
\label{mainframe.bottom}
\end{figure}

In the 
\helpfigref{bottom}{mainframe.bottom} there are 4 buttons 
(which may not necessarily be present in every folder)
and which trigger different actions, depending on the 
currently active folder:
\myitemize{
\item {\bf Display data}: opens a window with a table of data. 
\item {\bf Import}: allows to import data from other sources.
\item {\bf Export}: allows to export data as a single file.
\item {\bf Display graph}: opens a window with a graph of data .
}

Below there is the status bar, which will give you additional 
information about the current menu entry.

%%%%%%%%%%%%%%%%%%%%%%%%%%%%%%%%%%%%%%%%%%%%%%%%%%%%%%%%%%%%%%%%%%%%%%%%%%%%%%%
\subsection{Project}%
\label{project.management}

The main control for the project lies in the
\menu{File}. (See \helpfigref{File menu}{basic.file.menu}
\begin{figure}[h]
$$\image{0cm;0cm}{MFile.eps}$$%
\caption{The ``file'' menu}%
\label{basic.file.menu}
\end{figure}

There are 4 menu entries.
The \menuentry{New Project} erases the current project from memory after 
a user confirmation and sets all values to their defaults.

The \menuentry{Load Project} will load a project file, while the
\menuentry{Save Project} will save the project file with the same name, as
it had been loaded. If the project was started anew, the user 
will be prompted for a new filename.
Finally the \menuentry{Save Project as} allows to save the 
project with a different name.

The \menuentry{Quit} will exit \period.
If the Project has not been saved first,
the user will be asked for confirmation.


%%%%%%%%%%%%%%%%%%%%%%%%%%%%%%%%%%%%%%%%%%%%%%%%%%%%%%%%%%%%%%%%%%%%%%%%%%%%%%%
\subsection{Help menu}%
\label{basic.help}

The \menu{Help} gives the user additional information on
\period.

\begin{figure}[h]
$$\image{0cm;0cm}{MHelp.eps}$$%
\caption{The ``help'' menu}%
\label{basic.help.menu}
\end{figure}

There are a some menu entries:
\myitemize{
\item {\bf About} will inform about the version of \period.
\item {\bf Copyright} will start the help system and display
        the copyright message for the \period manual.
\item{\bf Introduction} will start the help system and display
        the Introduction section of the \period manual.
\item{\bf Contents} will start the help system and display
        the contents section of the \period manual.
\item {\bf Keyword search} will start the help system and 
        allow to search for a certain keyword in the \period manual.
}

%%%%%%%%%%%%%%%%%%%%%%%%%%%%%%%%%%%%%%%%%%%%%%%%%%%%%%%%%%%%%%%%%%%%%%%%%%%%%%%

\subsection{In case of a program error}%
\label{basic.error}

In case of a program error \period will try to recover from the
worst, inform the user that an fatal error has occurred and ask for a
filename to store the current project.

\note{A new filename is preferred, as the internal data structure may 
already have been destroyed and thus the output may 
be garbled.}

%%% Local Variables: 
%%% mode: latex
%%% TeX-master: "period98"
%%% End: 
